\section{Kernel Locking}

The kernel does not provide a global lock instead each set of functions that modify shared data and could be subject to concurrent invocation need to be protected through individual locks. The following generic rules apply to the kernel:
\begin{enumerate}
	\item during interrupt handling all calls up to \hyperref[kernel_2syscall_8c_a7af5fd6fbe3d25d1a6c23a7cfb8c5822]{ksc\_hdlr()} are executed with interrupts disabled
	\item \hyperref[kernel_2syscall_8c_a7af5fd6fbe3d25d1a6c23a7cfb8c5822]{ksc\_hdlr()} enabled interrupts, i.e. all system call handlers can be interrupted and might experience thread switches
	\item file system functions defined in \hyperref[fs_8c]{kernel/fs/fs.c}, except \hyperref[fs_8c_a11ce2a4ca94ef9167c296a3788c466a5]{fs\_register()}, are not thread-safe and calls to them need to be protected through \hyperref[fs_8c_a5d8afec6e07924b790d1de8c4b5e416a]{fs\_lock()} and \hyperref[fs_8c_a6becda9de89bcff22da1b19436174d13]{fs\_unlock()}
	\item file system system call handlers that modify the file system structure or rely on it, such as \hyperref[structfs__ops__t_a0ea05ccab180719ed0f32631d5ad71bf]{open()} and \hyperref[structfs__ops__t_a98fe7490086237888a48187e06214cc6]{rmnode()}, already acquire a lock through \hyperref[fs_8c_a5d8afec6e07924b790d1de8c4b5e416a]{fs\_lock()}. Since, the file system lock is not nestable respective callbacks for particular file system implementations are not allowed to acquire the file system lock again.
\end{enumerate}
