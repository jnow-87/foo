\chapter{Architecture Abstraction\label{chap:arch}}
	The architecture interface implements an abstraction layer for the actual target hardware. It provides the interface between the target hardware and the kernel, cf. Chapter~\ref{chap:kernel}.


\section{Interface Header}
	\begin{description}
	\item[include/arch/arch.h]
		\hyperref[arch_2arch_8h]{include/arch/arch.h} is the main architecture header file. It provides access to the architecture implementation defined through \lstinline{CONFIG_ARCH_HEADER}. The header includes the respective architecture header and provides macros to call kernel and common functions as well as a macro to access architecture information.	
		\lstinputlisting[linerange={11-11,16-16,19-19}, numbers=none]{include/arch/arch.h}

	\item[include/arch/types.h]
		\hyperref[arch_2types_8h]{include/arch/types.h} defines the types containing the architecture interface callback functions. That is \hyperref[structarch__callbacks__kernel__t]{arch\_callbacks\_kernel\_t} for kernel and \hyperref[structarch__callbacks__common__t]{arch\_callbacks\_common\_t} for common callbacks.

	\item[include/arch/*]
		The remaining header files in \lstinline{include/arch} contain wrapper macros to access all of the architecture interface functions separately.
	\end{description}


\section{Defining an Architecture}
	The following is required to implement support for a target architecture.

	\subsection{Types}
		\begin{itemize}
			\item \lstinline{struct arch_info_t}: struct with architecture specific data, such as clock frequencies
			\item \lstinline{struct thread_context_t}: struct to store the processor context
			\item \lstinline{enum int_type_t}: enum for interrupt types, e.g. debug, machine check
			\item \lstinline{enum int_num_t}: enum of available interrupt
			\item \lstinline{timebase_t}: basic timer type
		\end{itemize}

	\subsection{Callback Functions}
		The callback functions are registers by defining variables of type \hyperref[structarch__callbacks__kernel__t]{arch\_callbacks\_kernel\_t} and \hyperref[structarch__callbacks__common__t]{arch\_callbacks\_common\_t} named \lstinline{arch_cbs_kernel} and \lstinline{arch_cbs_common}. Further, in order to get access to the architecture specific data, a variable of type \lstinline{arch_info_t}, named \lstinline{arch_info} needs to be declared.

	\subsection{Macros}
		\begin{itemize}
			\item \lstinline{NINTERRUPTS}: number of interrupts
			\item \lstinline{TIMEBASE_INITIALISER}: init value for variable of type \lstinline{timebase_t}\\
		\end{itemize}

	\subsection{Kconfig}
		\begin{itemize}
			\item respective \quote{Processor type} has to be added
			\item the following CONFIG-variables need to be defined
				\begin{itemize}
					\item \lstinline{CONFIG_ARCH}
					\item \lstinline{CONFIG_PROCESSOR}
					\item \lstinline{CONFIG_ADDR_WIDTH_*}
					\item \lstinline{CONFIG_ARCH_HEADER}
				\end{itemize}
		\end{itemize}

	\subsection{Build System}
		\begin{itemize}
			\item libsys.o shall containing all objects that are required to be linked against applications
			\item linker script scripts/linker/kernel\_$<$arch$>$.lds
			\item memory layout print (optional): scripts/memlayout/main.c, if target specific sections are required
				\begin{itemize}
					\item mandatory memory sections shall be printed using \lstinline{PRINT_RANGE_EE}
					\item optional memory sections shall be printed using \lstinline{PRINT_RANGE}
				\end{itemize}

			\item memory layout check: scripts/memlayout/check.c, if target specific sections shall be checked add them to the \lstinline{regs} array.
		\end{itemize}

\vfill
\pagebreak
\section{Atmel \avr}
	This section summarises some critical specifics of the \avr architecture.

	\subsection{Linker sections}
		The \avr memory is split into code (flash) and data (\gls{sram}). Since the \gls{sram} is volatile memory all data that belong to linker sections such as .data and .bss are initially programmed to the flash memory and need to be copied to \gls{sram} during startup. The respective section offsets are defined in the linker scripts used to link the kernel (\fileref{scripts/linker/kernel\_avr.lds}). In case of the .data section, the base address within the flash is defined as \lstinline{__data_load_start} while the end address is defined as \lstinline{__data_load_end}. The respective target address in \gls{sram} are defined in \lstinline{__data_start} and \lstinline{__data_end}.

	\subsection{Interrupt Handling and Reset}
		At reset execution is started at the processor reset address. The location of the reset address is controlled through the \lstinline{BOOTRST} fuse and might be at the start of the application section (\lstinline{0x0}) or the boot loader. The start of the boot loader section is further controlled through the \lstinline{BOOTSZ0, BOOTSZ1} fuse bits. Depending on the memory configuration of the target controller the boot loader start address may vary.
		
		Depending on the fuse bit configuration the kernel base address has to be set through\\\lstinline{CONFIG_KERNEL_TEXT_BASE}. \remark{It shall be noted that the application flash is address is 2-byte chucks, hence addresses listed in the manual need to be multiplied by two in order to get the byte address.}

		The location of the interrupt vectors (except the reset vector) is controlled through \lstinline{MCUCR[IVSEL]}. They can either be placed at the start of the application section or the boot loader section. In the current implementation all interrupt vectors, including the reset vector, are mapped to the same location depending on \lstinline{CONFIG_KERNEL_TEXT_BASE}.

